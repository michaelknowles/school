\documentclass[12pt]{article}
\usepackage[margin=1in]{geometry} 
\usepackage{setspace}
\usepackage{fancyhdr} % heacer
\pagestyle{fancy}
%\fancyhf{}
\fancyhead[l,lo]{Michael Knowles} % top left header
\fancyhead[r,ro]{\textit{Journal}} %top right header
\doublespacing
\begin{document}
\paragraph{Why is identity management important? How do you manage your identities?}
Identity Management - ``The communication strategies that people use to influence how others view them.''
\newline
\indent Identity management is important because it allows people to tailor their personality to their intended audience.  One of the most prominent examples of this can be found in many politicians.  Recently, four Republicans have resigned in Oklahoma after being caught doing things very differently from what their public personas would do.  One has been charged with sexual battery, one was taking upskirt pictures of women, one was part of child sex trafficking, and another sexually harrassed former assistants.  Of course, this came as a surprise to many people because they had previously thought of these men in a completely different light.  
\newline
\indent This also perfectly highlights the two ``selves'' that people have:  The \emph{perceived} self - who a person believes themselves to be, and the \emph{presenting} self - how a person wants others to view them.  In these cases, it's very likely that these mens' two different selves were very different from each other.  Within each of these two categories, it's very likely that there are many different selves, or identities, that people have.  For example, people may present themselves differently to different groups.
\newline
\indent People may read this and think that identity management is all about being ``fake'' or being a ``chameleon''; however, that's not at all the case.  It's something that everybody does, whether they realize it or not.  Experts have found several ways that people manage their identities: to start and manage relationships, to gain compliance of others, to save others' face, and to explore new selves.  People have done all of these throughout their lives, and it shows that identity management is a very important skill to have.
\newline
\indent First, without identity management, relationships would be extremely difficult.  For example, when meeting new people, it's customary to smile, appear kind and happy, and shake their hands.  This is regardless of the fact that you may hate smiling and shaking hands, or that you actually feel angry about something unrelated.  Everybody learns the importance of first impressions, because they want others' to start with a positive reaction.
\newline
\indent Second, without identity management, it would be hard to persuade others.  Many people can think of a time in their lives when they did something just for appearances.  For example, maybe they like to be naked, but they'll put on clothes when they go out.  Women will wear high heeled shoes even though it hurts them.  Men will wear undershirts, shirts, and suit jackets even though it's too hot for all of that.  If they weren't dressed professionaly, it may have made their lives difficult at work or school.
\newline
\indent Third, without identity management, people wouldn't be able to interact amicably with each other.  Empathy is key to interacting with society.   Manners that are considered required aren't always natural.  Sometimes people are told they should speak their mind, but the understood exception is generally that it shouldn't be at the expense of others.  For example, before children learn this exception, they might be heard making rude comments about another person.  Soon, once they learn about identity management, they understand that they shouldn't say certain things or that they should keep certain things to themselves.
\newline
\indent Fourth, without identity management, people would have difficulty learning about themselves.  This is probably most visible in children when some go through different ``phases'' in their lives.  Parents may notice that their children will behave one way when they're young, another in their teenage years, and maybe several more ways into their adulthood.  People may also use this as a way to improve themselves.  For example, maybe they don't like their \emph{perceived} self and want to manage their identities to create seomthing they approve of.
\newline
\indent As mentioned before, identity management can occur whether one consciously realizes it or not.  Throughout reading this chapter, I've recognized that a good amount of my behavior changes depending on my mood, my feeling, who I'm interacting with, the setting, and more.  One of the most noticeable changes I make is my appearance.  I love to be comfortable.  However, if I'm going to work or an interview, I will dress very uncomfortably sometimes to present myself in a certain fashion to the people at these settings.  Even when there isn't a dress code, I'll find that I dress in ways that I wouldn't if I was alone.
\newline
\indent Sometimes, my identity changes and I don't even want it to.  A lot of times I want to present myself in a certain way, but am unable to for a reason I don't even know.  For example, there are some people I meet who I just click with.  I find that when I am able to interact with them naturally and able to make a good impression.  Other people I'll meet and be the most awkward, shy person imaginable.  As the book notes, identity management is collaborative.  It's likely that  I'm picking up cues from the other party that my conscious mind doesn't really register, even though it has a conscious effect on me. 
\end{document}